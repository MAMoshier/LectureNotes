\def \Topic {Sets and Functions}
\def \Author {M. Andrew Moshier}
\def \Date {October 2014}
\def \Course {Discrete Mathematics}
\long \def \Overview { The mathematical universe consists of various things: numbers, functions, graphs, lists and so on. A \emph {set} is a collection of things. For example, the collection of all natural numbers is a set. A \emph {function} is a correlation of the members of one set with members of another set. These two abstract concepts (sets and functions) form a conceptual framework in which virtually all of mathematics can be built. So an understanding of sets and functions is key to a rigorous approach to most other parts of mathematics. This conceptual framework can itself be put on a formal, precise footing called the Category of Sets and Functions. \par In these lectures, we build up the Category of Sets and Functions, so that we can use these things as the basic building blocks of everything else we do. }
