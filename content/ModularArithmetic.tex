\chapter{Modular Arithmetic}

\begin{goals} 
	\begin{goal}{Lecture}
		\item Introduce counting ``on a dial''
		\item Extend to arithemetic.
		\item Introduce the idea of an \emph{equivalence relation} via modular equivalence.
		\item Introduce the extended Euclidean algorithm.
	\end{goal}
	
	\begin{goal}{Study}
		\item Learn to calculate within a modulus.
		\item Learn to use the extended Euclidean algorithm to determine multipicative inverses.
	\end{goal}
\end{goals}

The positions on a dial such as a clock have some of the features of a number system. There is a ``top of the dial''. Each position has a next position (clock-wise) and a previous position (counter-clockwise). As it happens, arithmetic on a dial, which mathemticians call \emph{modular arithmetic}, is fundamental to how we actually write numbers, for example as base 10 numerals, and how we perform arithmetic on those numerals. Cryptography is also an important area of application.  
 
Recall that the postulates of the natural numbers included the requirement that $0$ does not have  a predecessor (Postulate \ref{lec:NaturalNumbers}.\ref{post:NatZero}).  Exercise \ref{lec:NaturalNumbers}.\ref{exer:CyclicModel} asked you to draw depictions in which Postulate \ref{lec:NaturalNumbers}.\ref{post:NatZero} is relaxed. 
Assuming your response to Exercise \ref{lec:NaturalNumbers}.\ref{exer:CyclicModel} is correct, you may have drawn something similar to the dial-like structures in Figure \ref{fig:ModCycles}.

In the integers, $0$ (as well as all other integers) has a precedessor. Suppose we \emph{replace} Postulate \ref{lec:NaturalNumbers}.\ref{post:NatZero} with the requirement that every value has a predecessor. 
Then the integers satisfy all postulates, including the requirement that nothing can be removed. But it is also true that the cyclical pictures of Figure \ref{fig:ModCycles} do the job. 

For now, in order to avoid confusing pictures like this with the natural numbers, let us use alternative notation for ``the start'' and ``successor'', which we have been denoting as $0$ and $n^\nxt$ in the natural numbers.
For structures like those in \ref{fig:ModCycles}, let us write $\dialtop$ for the ``top of the dial'' and $\nx(a)$ for the ``next'' value after $a$ in the clock-wise direction. So we are now considering structures of the following sort.

\begin{defn}
	A \newterm{dial} is a structure satisfying the following conditions.
\begin{enumerate}
	\item There is a value $\dialtop$ and every value $a$ has a successor $\nx(a)$.
	\item Every value has a predecessor: for every $b$, there is an $a$ so that $\nx(a) = b$.
	\item Predecessors are unique: $\nx(a) = \nx(b)$ implies $a=b$.
	\item No values can be removed without violating 1 and 2.
\end{enumerate}
\end{defn}

\begin{example}
The integers form a dial by letting $\dialtop={}^+0$ and $\nx(a)=a+1$.
Every integer has a successor. Every integer has a unique predecessor. Moreover, if we were to remove any integers, then some other remaining integers would either not have successors or not have predecessors.

Any actual cyclic structure like those in Figure \ref{fig:ModCycles} is a dial.

Consider a dial with labels ``$D_0$'' through ``$D_9$'' printed on its edge. Then the label clockwise after ``$D_9$'' is ``$D_0$''. So letting $\dialtop = D_0$, this is a physical model of $\ZZ_{10}$.
\end{example}

\begin{exer}
	\begin{exercise}
		\item A standard wall clock depicts $Z_m$ for what $m$? What changes on the labels of a clock dial would you make to highlight this fact?
	\end{exercise}
\end{exer}   

Almost the same definitions for addition and multiplication work in dial as they did in the natural numbers:

\begin{defn}[Addition on a Dial]
	Suppose $\DD$ is a dial (it satisfies the four postulates above). For any $a$ in $\DD$ and any natural number $n$, the \newterm{sum} $a+_\DD n$ is given by the equations:
	\begin{align*}
		a +_\DD 0 &= a\\
		a +_\DD k^\nxt &= \nx(a+_\DD k)
	\end{align*}
\end{defn}

\begin{lem}
	Let $\DD$ be a dial.
	For any $a$ in $\DD$ and any natural numbers $n$ and $p$, $a +_\DD (n+p) = (a +_\DD n) +_\DD p$.
	
	\begin{proof}
		The proof is by induction on $p$. 
		\begin{itemize}
			\item{}[Basis] $a +_\DD (n+0) = a +_\DD n = (a +_\DD n) +_\DD 0$.
			\item{}[Inductive Hypothesis] Assume that $a +_\DD (n+k) = (a +_\DD n) +_\DD k$ for some $k$.
			\item{}[Inductive Step] 
			\begin{align*}
				a +_\DD (n+k^\nxt) &= a +_\DD (n+k)^\nxt\\ 
				                 &= \nx(a +_\DD (n+k))\\
				                 &= \nx((a +_\DD n) +_\DD k)\\
				                 &= (a +_\DD n) +_\DD k^\nxt.
			\end{align*}
		\end{itemize}
	\end{proof}
\end{lem}

Suppose we have a dial. How can we determine that it is $\ZZ_m$ and not the integers? Suppose there is some $n$ for which $\dialtop +_\DD n = \dialtop$. Then there is a smallest one $m$. In that case, $m$ is called the \newterm{modulus} and the dial is denoted by $\ZZ_m$. If no such natural number exists, the dial is a model of the integers (that is, it satisfies all the requirements laid out in Lecture \ref{lec:OtherNumbers})and is denoted by $\ZZ$.

In case we are interested in a modular dial (one of the form $\ZZ_m$) we will write $+_m$ instead of $+_{\ZZ_m}$ to avoid clutter.

We may extend addition in any dial to all integers because every element has a unique predecessor. That is, we may write $a-_\DD 1$ for the predecessor of $a$, $a-_\DD 2$ for the predecessor of $a-_\DD 1$ and so on. In general, $a-n$ is the unique value on the dial satisfying $(a-\DD n) +_\DD n = a$. Again, in $\ZZ_m$, we write $-_m$ in place of $-{\ZZ_m}$. 

By the principle of induction, in any modulus, every $a$ in $\ZZ_m$ can be written uniquely as $\dialtop + k$ for some $k<m$.
That is, we can easily show that the values of the form $\dialtop + k$ constitute a structure satisfying all the postulates. So by minimality, no values can be removed. Furthermore, by induction, if $k,j<m$ and $\dialtop +_m k = \dialtop +_m j$, then $k=j$. 

\begin{lem}
	Fix a modulus $m$. 
	For any $a$ in $\ZZ_m$, there is a natural number $k<m$ so that $a +_m k = T_0$. 	
	\begin{proof}
		For $a = T_0 +_m j$, we take $k = m-j$. Then $a +_m k = (T_0 +_m j) + (m-j)) = T_0 +_m m = T_0$.
	\end{proof}
\end{lem}

With these observations, we can define addition in $\ZZ_m$ directly by $a +_m (\dialtop+_m k) = a +_m k$. 

\begin{exer}
	\begin{exercise}
		\item Calculate the following values step-by-step.
		\begin{enumerate}
			\item $D_4 +_{5}4$
			\item $D_8+_{10}3$
			\item $(D_2 +_{7} 3) +_{7} D_2$
		\end{enumerate}
		\item Show that addition in $\ZZ_m$ is commutative, associative and has an identity element. Furthermore, show that every element of $\ZZ_m$ has an additive inverse.
		\item Prove that for any $a$ in $\ZZ_m$, it is the case that $a +_m m = a$.
	\end{exercise}
\end{exer}

We may now also define multiplication in $\ZZ_m$.

\begin{defn}
	For element $a$ of $\ZZ_m$ and natural number $n$, define $a\cdot_m n$ b the equations
	\begin{align*}
		a\cdot_m  0 &= dialtop\\
		a\cdot_m k^\nxt &= a +_m a\cdot_m k
	\end{align*}
	
	Also define multiplication of elements of $\ZZ_{m}$ by $a\cdot_m D_k = a\cdot_m k$
\end{defn}


\begin{lem}\label{lem:modular-mult}
	For any positive natural number $m$, $a$ in $\ZZ_m$ and natural numbers $n$ and $p$, it is the case that 
	\begin{align*} 
		(a \cdot_m n)+_m (a\cdot_m p) &= a\cdot_m (n+p)\\
		(a\cdot_m n)\cdot_m p &= a\cdot_m (np).
	\end{align*}

\begin{proof}
	Exercise.
\end{proof}
\end{lem}

\begin{exer}
	\begin{exercise}
		\item Prove Lemma \ref{lem:modular-mult}.
	\end{exercise}
\end{exer}




