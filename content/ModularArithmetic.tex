\chapter{Modular Arithmetic}

\begin{goals}
	\begin{goal}{Lecture}
		\item Introduce counting ``on a dial''
		\item Extend to arithemetic.
		\item Introduce the idea of an \emph{equivalence relation} via modular equivalence.
		\item Introduce the extended Euclidean algorithm.
	\end{goal}
	
	\begin{goal}{Study}
		\item Learn to calculate within a modulus.
		\item Learn to use the extended Euclidean algorithm to determine multipicative inverses.
	\end{goal}
\end{goals}

Recall the postulates of the natural numbers included the requirement that $0$ does not have  a predecessor (Postulate \ref{lec:NaturalNumbers}.\ref{post:NatZero}).  Exercise \ref{lec:NaturalNumbers}.\ref{exer:CyclicModel} asked you to draw depictions in which Postulate \ref{lec:NaturalNumbers}.\ref{post:NatZero} is relaxed. 
Assuming your response to Exercise \ref{lec:NaturalNumbers}.\ref{exer:CyclicModel} is correct, you may have drawn something similar to the structures in Figure \ref{fig:ModCycles}.

THe integers clearly also fail to have this property because $0$ (as well as all other integers) has a precedessor. Suppose we \emph{replace} Postulate \ref{lec:NaturalNumbers}.\ref{post:NatZero} with the the requirement that every value has a predecessor. Then the integers satisfy all postulates, including Induction. But it is also true that the cyclical pictures of Figure \ref{fig:ModCycles} do the job. 

For now, in order to avoid confusing pictures like this with the natural numbers, let us use alternative notation for ``the start'' and ``successor'', which we have been denoting as $0$ and $n^\nxt$ in the natural numbers.
For structures like those in \ref{fig:ModCycles}, let us write $T_0$ for the ``top'' of the cycle and $\nx(a)$ for the ``next'' value after $a$. So we are now considering structures of the following sort.

\begin{enumerate}
	\item There is a value $T_0$ and every value $a$ has a successor $\nx(a)$.
	\item Every value has a predecessor: for every $b$, there is an $a$ so that $\nx(a) = b$.
	\item Predecessors are unique: $\nx(a) = \nx(b)$ implies $a=b$.
	\item No values can be removed without violating 1.
\end{enumerate}

With a bit of effort, one can show that the \emph{only} structures that satisfy these four postulates are either (i) structures that look exactly like the integers and (ii) the cyclic structures that differ from Figure \ref{fig:ModCycles} only by the number of elements in the picture (and the names we give the elements). Indeed for every natural number $m>0$, there is a cycle consisting of $m$ elements.

\begin{defn}
	We refer to the structure of the integers as $\ZZ$. We refer to a cyclic structure with $m$ elements as $\ZZ_m$. For $\ZZ_m$ the natural number $m$ is called the \newterm{modulus}.
\end{defn}

We wish to investigate how $\ZZ_m$ is similar to $\ZZ$. 

For starters, notice that we can give standard names for elements of $\ZZ_m$. Let us refer to them as $T_0$, $T_1$, \ldots, $T{m-1}$. So $\nxt(T_{m-1}) = T_0$. This is what makes the whole this cyclic.
Think of a dial with labels ``$T_0$'' through ``$T_9$'' printed on its edge. Then the label after ``$T_9$'' is ``$T_0$''. So this dial is a good picture of $\ZZ_{10}$. Indeed, dials with $m$ labels are exactly the right physical models of $\ZZ_m$. 

\begin{exer}
	\begin{exercise}
		\item A standard wall clock depicts $Z_n$ for what $n$? What changes on the labels of a clock dial would you make to highlight this fact?
	\end{exercise}
\end{exer}   

The same definitions for addition and multiplication work in any $\ZZ_m$ as they did in the natural numbers:

\begin{defn}[Modular Addition in $\ZZ_m$]
	Fix a positive natural number $m$. Define addition in $\ZZ_m$ as follows: For any $a$ in $\ZZ_m$ and any natural number $n$, the sum $a+_m n$ is given by the equations:
	\begin{align*}
		a +_m 0 &= a\\
		a +_m k^\nxt &= \nx(a+_m k)
	\end{align*}
\end{defn}

\begin{lem}
	For any $a$ in $\ZZ_m$ and any natural numbers $n$ and $p$, $a +_m (n+p) = (a +_m n) +_m p$.
	
	\begin{proof}
		The proof is by induction on $p$. 
		\begin{itemize}
			\item{}[Basis] $a +_m (n+0) = a +_m n = (a +_m n) +_m 0$.
			\item{}[Inductive Hypothesis] Assume that $a +_m (n+k) = (a +_m n) +_m k$ for some $k$.
			\item{}[Inductive Step] 
			\begin{align*}
				a +_m (n+k^\nxt) &= a +_m (n+k)^\nxt\\ 
				                 &= \nx(a +_m (n+k))\\
				                 &= \nx((a +_m n) +_m k)\\
				                 &= (a +_m n) +_m k^\nxt.
			\end{align*}
		\end{itemize}
	\end{proof}
\end{lem}

Notice that $a +_m (k + nm)= a +_m (k+pm)$ for any $a$ in $\ZZ_m$ and any natural numbers $k$, $n$ and $p$. 
This is because counting forward by $m$  leads back to the original position, so counting forward by $2m$, $3m$ and so on also leads back to the original position. 
Also, we may extend addition to all integers because of the following lemma.
By the principle of induction, every $a$ in $\ZZ_m$ can be written uniquely as $T_0 + k$ for some $k<m$.
That is, we can easily show that the values of the form $T_0 + k$ constitute a structure in satisfying all the postulates. So by minimality, no values can be removed. Furthermore, by induction, if $k,j<m$ and $T_0 +_m k = T_0 +_m j$, then $k=j$. 

\begin{lem}
	For any $a$ in $\ZZ_m$, there is a natural number $k<m$ so that $a +_m k = T_0$. 	
	\begin{proof}
		For $a = T_0 +_m j$, we take $k = m-j$. Then $a +_m k = (T_0 +_m j) + (m-j)) = T_0 +_m m = T_0$.
	\end{proof}
\end{lem}

With these observations, we can define addition in $\ZZ_m$ directly by $T_n +_m T_p = T_k +_m p$. 

\begin{exer}
	\begin{exercise}
		\item Calculate the following values step-by-step.
		\begin{enumerate}
			\item $T_4 +_{5}} 4$
			\item $T_8+_{10}3$
			\item $(T_2 +_{7} 3) +_{7} T_2$
		\end{enumerate}
		\item Show that addition in $\ZZ_m$ is commutative, associative and has an identity element. Furthermore, show that every element of $\ZZ_m$ has an additive inverse.
	\end{exercise}
\end{exer}

We may now also define multiplication in $\ZZ_m$.

\begin{defn}
	For element $a$ of $\ZZ_m$ and natural number $n$, define $a\cdot n$ b the equations
	\begin{align*}
		a\cdot_m  0 &= T_0\\
		a\cdot_m k^\nxt &= a +_m a\cdot_m k
	\end{align*}
	
	Also define multiplication of elements of $\ZZ_{m}$ by $T_n \cdot_m T_p = T_n\cdot_m p$
\end{defn}


\begin{lem}\label{lem:modular-mult}
	For any positive natural number $m$, $a$ in $\ZZ_m$ and natural numbers $n$ and $p$, it is the case that 
	\begin{align*} 
		(a \cdot_m n)+_m (a\cdot_m p) &= a\cdot_m (n+p)\\
		(a\cdot_m n)\cdot_n p &= a\cdot_m (np).
	\end{align*}

\begin{proof}
	Exercise.
\end{proof}
\end{lem}

\begin{exer}
	\begin{exercise}
		\item Prove Lemma \ref{lem:modular-mult}.
	\end{exercise}
\end{exer}


