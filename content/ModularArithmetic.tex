\chapter{Modular Arithmetic}

\begin{goals}
	\begin{goal}{Lecture}
		\item Introduce counting ``on a dial''
		\item Extend to arithemetic.
		\item Introduce the idea of an \emph{equivalence relation} via modular equivalence.
		\item Introduce the extended Euclidean algorithm.
	\end{goal}
	
	\begin{goal}{Study}
		\item Learn to calculate within a modulus.
		\item Learn to use the extended Euclidean algorithm to determine multipicative inverses.
	\end{goal}
\end{goals}

Recall the postulates of the natural numbers included the requirement that $0$ does not have  a predecessor (Postulate \ref{lec:NaturalNumbers}.\ref{post:NatZero}).  Exercise \ref{lec:NaturalNumbers}.\ref{exer:CyclicModel} asked you to draw depictions in which Postulate \ref{lec:NaturalNumbers}.\ref{post:NatZero} is relaxed. 
Assuming your response to Exercise \ref{lec:NaturalNumbers}.\ref{exer:CyclicModel} is correct, you may have drawn something similar to the structures in Figure \ref{fig:ModCycles}.

THe integers clearly also fail to have this property because $0$ (as well as all other integers) has a precedessor. Suppose we \emph{replace} Postulate \ref{lec:NaturalNumbers}.\ref{post:NatZero} with the the requirement that every value has a predecessor. Then the integers satisfy all postulates, including Induction. But it is also true that the cyclical pictures of Figure \ref{fig:ModCycles} do the job. 

For now, in order to avoid confusing pictures like this with the natural numbers, let us use alternative notation for ``the start'' and ``successor'', which we have been denoting as $0$ and $n^\nxt$ in the natural numbers.
For structures like those in \ref{fig:ModCycles}, let us write $T$ for the ``top'' of the cycle and $\next(a)$ for the ``next'' value after $a$. So we are now considering structures of the following sort.

\begin{enumerate}
	\item There is a value $T$ and every value $a$ has a successor $\next(a)$.
	\item Every value has a predecessor: for every $b$, there is an $a$ so that $\next(a) = b$.
	\item Predecessors are unique: $\next(a) = \next(b)$ implies $a=b$.
	\item No values can be removed without violating 1.
\end{enumerate}

With a bit of effort, one can show that the \emph{only} structures that satisfy these four postulates are either (i) structures that look exactly like the integers and (ii) the cyclic structures that differ from Figure \ref{fig:ModCycles} only by the number of elements in the picture (and the names we give the elements). Indeed for every natural number $m>0$, there is a cycle consisting of $m$ elements.

\begin{defn}
	We refer to the structure of the integers as $\ZZ$. We refer to a cyclic structure with $m$ elements as $\ZZ_m$. 
\end{defn}

We wish to investigate how $\ZZ_m$ is similar to $\ZZ$. 

For starters, notice that we can give standard names for elements of $\ZZ_m$. Let us refer to them as $T_0$, $T_1$, \ldots, $T{m-1}$. To $T=T_0$ and $\nxt(T_{m-1}) = T_0$. This is what makes the whole this cyclic.
Think of a dial with labels ``$T_0$'' through ``$T_9$'' printed on its edge. Then the label after ``$T_9$'' is ``$T_0$''. So this dial is a good picture of $\ZZ_{10}$. Indeed, dials with $m$ labels are exactly the right physical models of $\ZZ_m$. 

\begin{exer}
	\begin{exercise}
		\item A standard wall clock depicts $Z_n$ for what $n$? What changes on the labels of a clock dial would you make to highlight this fact?
	\end{exercise}
\end{exer}   


The same definitions for addition and multiplication work in any $\ZZ_m$ as they did in the natural numbers:

\begin{defn}[Modular Arithmetic in $\ZZ_m$]
	Fix a positive natural number $m$. Define addition and multiplication in $\ZZ_m$ as follows: For any $a$ in $\ZZ_m$ and any natural number $n$, the sum $a+n$ and product $a\cdot n$ is given by the equations:
	\begin{align*}
		a + 0 &= a\\
		a + k^\nxt &= \next(a+k)\\
		a\cdot 0 &= 0\\
		a\cdot k^\nxt &= a + a\cdot k
	\end{align*}
\end{defn}

This definition is subtle because plus sign and times sign do not refer to addition and multiplication on the natural numbers, or even in the integers. They denote new operations defined on the elements of $\ZZ_m$. So actually, we could be decorating these symbols to highlight that fact. That is, we could write $a+_m n$ and $a\cdot_m n$ to emphasize that $+_m$ and $\cdot_m$ are operations defined only in $\ZZ_m$. You will probably agree that that extra bit of notation would only clutter things, so we omit it.

\begin{lem}
	For any positive natural number $m$, and $a$ in $\ZZ_m$ 
\end{lem}
