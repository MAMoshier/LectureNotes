\chapter{Modular Arithmetic}

\begin{goals}
	\begin{goal}{Lecture}
		\item Introduce counting ``on a dial''
		\item Extend to arithemetic.
		\item Introduce the idea of an \emph{equivalence relation} via modular equivalence.
		\item Introduce the extended Euclidean algorithm.
	\end{goal}
	
	\begin{goal}{Study}
		\item Learn to calculate within a modulus.
		\item Learn to use the extended Euclidean algorithm to determine multipicative inverses.
	\end{goal}
\end{goals}

Recall the postulates of the natural numbers included the requirement that $0$ does not have  a predecessor (Postulate \ref{lec:NaturalNumbers}.\ref{post:NatZero}).  Exercise \ref{lec:NaturalNumbers}.\ref{exer:CyclicModel} asked you to draw depictions in which Postulate \ref{lec:NaturalNumbers}.\ref{post:NatZero} is relaxed. 
Assuming your response to Exercise \ref{lec:NaturalNumbers}.\ref{exer:CyclicModel} is correct, you may have drawn something similar to the structures in Figure \ref{fig:ModCycles}.

The integers clearly also fail to have this property because $0$ (as well as all other integers) has a precedessor. Suppose we \emph{replace} Postulate \ref{lec:NaturalNumbers}.\ref{post:NatZero} with the requirement that every value has a predecessor. Then the integers satisfy all postulates, including Induction. But it is also true that the cyclical pictures of Figure \ref{fig:ModCycles} do the job. 

For now, in order to avoid confusing pictures like this with the natural numbers, let us use alternative notation for ``the start'' and ``successor'', which we have been denoting as $0$ and $n^\nxt$ in the natural numbers.
For structures like those in \ref{fig:ModCycles}, let us write $D_0$ for the ``top of the dial'' and $\nx(a)$ for the ``next'' value after $a$ in the clock-wise direction. So we are now considering structures of the following sort.

\begin{defn}
	A \emph{dial} is a structure satisfying the following conditions.
\begin{enumerate}
	\item There is a value $D_0$ and every value $a$ has a successor $\nx(a)$.
	\item Every value has a predecessor: for every $b$, there is an $a$ so that $\nx(a) = b$.
	\item Predecessors are unique: $\nx(a) = \nx(b)$ implies $a=b$.
	\item No values can be removed without violating 1.
\end{enumerate}
\end{defn}

With a bit of effort, one can show that the \emph{only} structures that satisfy these four postulates are either (i) structures that look exactly like the integers and (ii) the cyclic structures that differ from Figure \ref{fig:ModCycles} only by the number of elements in the picture (and the names we give the elements). Indeed for every natural number $m>0$, there is a dial consisting of $m$ elements.

\begin{defn}
	We refer to the structure of the integers as $\ZZ$. We refer to a cyclic structure with $m$ elements as $\ZZ_m$. For $\ZZ_m$ the natural number $m$ is called the \newterm{modulus}.
\end{defn}

We wish to investigate how the dials $\ZZ_m$ are similar to $\ZZ$. 

For starters, we can informally give standard names for elements of $\ZZ_m$. Let us refer to them as $D_0$, $D_1$, \ldots, $D_{m-1}$. So $\nx(D_{m-1}) = D_0$. This is what makes the whole this cyclic. Notice, however, that $D_1$, $D_2$ and so on are really just an abbreviations for $\nx(D_0)$, $\nx(nx(D_0))$ and so on,  just like $1$, $2$ and so on are an abbreviations (in our notation for natural numbers) for $0^\nxt$, $0^{\nxt\nxt}$ and so on.

\begin{example}
Consider a dial with labels ``$D_0$'' through ``$D_9$'' printed on its edge. Then the label clockwise after ``$D_9$'' is ``$D_0$''. This is a good picture of $\ZZ_{10}$. Indeed, actual physical dials with $m$ evenly spaced labels are exactly the right physical models of $\ZZ_m$.
\end{example}

\begin{exer}
	\begin{exercise}
		\item A standard wall clock depicts $Z_m$ for what $m$? What changes on the labels of a clock dial would you make to highlight this fact?
	\end{exercise}
\end{exer}   

Almost same definitions for addition and multiplication work in any $\ZZ_m$ as they did in the natural numbers:

\begin{defn}[Addition on a Dial]
	Suppose $\DD$ is a dial (it satisfies the four postulates above). For any $a$ in $\DD$ and any natural number $n$, the \newterm{sum} $a+_\DD n$ is given by the equations:
	\begin{align*}
		a +_\DD 0 &= a\\
		a +_\DD k^\nxt &= \nx(a+_\DD k)
	\end{align*}
\end{defn}

\begin{lem}
	Let $\DD$ be a dial.
	For any $a$ in $\DD$ and any natural numbers $n$ and $p$, $a +_\DD (n+p) = (a +_\DD n) +_\DD p$.
	
	\begin{proof}
		The proof is by induction on $p$. 
		\begin{itemize}
			\item{}[Basis] $a +_\DD (n+0) = a +_\DD n = (a +_\DD n) +_\DD 0$.
			\item{}[Inductive Hypothesis] Assume that $a +_\DD (n+k) = (a +_\DD n) +_\DD k$ for some $k$.
			\item{}[Inductive Step] 
			\begin{align*}
				a +_\DD (n+k^\nxt) &= a +_\DD (n+k)^\nxt\\ 
				                 &= \nx(a +_\DD (n+k))\\
				                 &= \nx((a +_\DD n) +_\DD k)\\
				                 &= (a +_\DD n) +_\DD k^\nxt.
			\end{align*}
		\end{itemize}
	\end{proof}
\end{lem}

Suppose we have a dial. How can we determine that it is $\ZZ_m$ and not the integers? Suppose there is some $n$ for which $D_0 +_\DD n = D_0$. Then there is a smallest one $m$. In that case, $m$ is called the \newterm{modulus} and the dial is denoted by $\ZZ_m$. If no such natural number exists, the dial is a model of the integers (that is, it satisfies all the requirements laid out in Lecture \ref{lec:OtherNumbers})and is denoted by $\ZZ$.

In case we are interested in a modular dial (one of the form $\ZZ_m$) we will write $+_m$ instead of $+_{\ZZ_m}$ to avoid clutter.

We may extend addition in any dial to all integers because every element has a unique predecessor. That is, we may write $a-_\DD 1$ for the predecessor of $a$, $a-_\DD 2$ for the predecessor of $a-_\DD 1$ and so on. In general, $a-n$ is the unique value on the dial satisfying $(a-\DD n) +_\DD n = a$. Again, in $\ZZ_m$, we write $-_m$ in place of $-{\ZZ_m}$. 

By the principle of induction, in any modulus, every $a$ in $\ZZ_m$ can be written uniquely as $D_0 + k$ for some $k<m$.
That is, we can easily show that the values of the form $D_0 + k$ constitute a structure satisfying all the postulates. So by minimality, no values can be removed. Furthermore, by induction, if $k,j<m$ and $D_0 +_m k = D_0 +_m j$, then $k=j$. 

\begin{lem}
	Fix a modulus $m$. 
	For any $a$ in $\ZZ_m$, there is a natural number $k<m$ so that $a +_m k = T_0$. 	
	\begin{proof}
		For $a = T_0 +_m j$, we take $k = m-j$. Then $a +_m k = (T_0 +_m j) + (m-j)) = T_0 +_m m = T_0$.
	\end{proof}
\end{lem}

With these observations, we can define addition in $\ZZ_m$ directly by $T_n +_m T_p = T_k +_m p$. 

\begin{exer}
	\begin{exercise}
		\item Calculate the following values step-by-step.
		\begin{enumerate}
			\item $T_4 +_{5}} 4$
			\item $T_8+_{10}3$
			\item $(T_2 +_{7} 3) +_{7} T_2$
		\end{enumerate}
		\item Show that addition in $\ZZ_m$ is commutative, associative and has an identity element. Furthermore, show that every element of $\ZZ_m$ has an additive inverse.
		\item Prove that for any $a$ in $\ZZ_m$, it is the case that $a +_m m = a$.
	\end{exercise}
\end{exer}

We may now also define multiplication in $\ZZ_m$.

\begin{defn}
	For element $a$ of $\ZZ_m$ and natural number $n$, define $a\cdot n$ b the equations
	\begin{align*}
		a\cdot_m  0 &= T_0\\
		a\cdot_m k^\nxt &= a +_m a\cdot_m k
	\end{align*}
	
	Also define multiplication of elements of $\ZZ_{m}$ by $T_n \cdot_m T_p = T_n\cdot_m p$
\end{defn}


\begin{lem}\label{lem:modular-mult}
	For any positive natural number $m$, $a$ in $\ZZ_m$ and natural numbers $n$ and $p$, it is the case that 
	\begin{align*} 
		(a \cdot_m n)+_m (a\cdot_m p) &= a\cdot_m (n+p)\\
		(a\cdot_m n)\cdot_n p &= a\cdot_m (np).
	\end{align*}

\begin{proof}
	Exercise.
\end{proof}
\end{lem}

\begin{exer}
	\begin{exercise}
		\item Prove Lemma \ref{lem:modular-mult}.
	\end{exercise}
\end{exer}




