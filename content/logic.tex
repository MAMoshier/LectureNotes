\chapter{Proofs}

\begin{goals}
	\begin{goal}{Lecture}
		\item Introduce basic techniques of mathematical proofs.
		\item Give representative examples of the techniques.
		\item Discuss ``style'' and prosody of proofs.
	\end{goal}
	
	\begin{goal}{Study}
		\item Be able to recognize general types of proof.
		\item Begin to be able to construct simple proofs.
		\item Learn to avoid common mistakes in your proofs.
	\end{goal}
\end{goals}

\section{Statements}

The object of a proof is a \newterm{statement}, such as ``for all $m$ and for all $n$, it is the case that $m+n=n+m$''.
Statements are, roughly speaking,  declarative sentences. 
Of course, a statement may involve mathematical notation, but it will nevertheless make a propositional claim that such and such is true.

\begin{shout}
	You should always insist that a statement can be read aloud as a grammatical sentence.
\end{shout}

\begin{example}
	Here are some examples of statements.
	\begin{itemize}
		\item $9$ is an integer.
		\item $4$ is a perfect square.
		\item $7$ is even.
		\item Addition is commutative.
		\item Every natural number has a successor.
		\item For every integer $n$, $3+n=7$.
	\end{itemize}
	Some of these are true, some are not. But the point is that they are the sorts of sentences that can be judged to be one or the other.
	
	Here are some non-statements.
	\begin{itemize}
		\item $5+8$ -- not making an assertion.
		\item $3+x=7$ -- not a statement because we do not know what $x$ stands for.
	\end{itemize} 
\end{example}

Statments can be combined by \emph{logical connectives} to obtain compound statements, the meanings of which are mostly obvious. If $P$ and $Q$ are statments, then so are each of the following:
\begin{itemize}
	\item $P$ and $Q$.
	\item $P$ or $Q$.  
	\item $P$ implies $Q$.
	\item $P$ if and only if $Q$.
	\item It is not the case that $P$.
	\item $P$, but not $Q$.  
\end{itemize}
We can also write ``If $P$ then $Q$'' or ``$Q$, whenever $P$'' or similar formulations in place of ``$P$ implies $Q$''. The phrase ``if and only if'' is often abbreviated ``iff'' (but it is pronounced as ``if and only if'').

Two main things we need to understand about these compound statements are (i) how to prove a compound statement and (ii) how to use a compound statement in proving some other statement. 
Here is an example of use. 
Suppose we know that ``$P$ implies $Q$'' (perhaps we know a proof of it), and suppose we also know $P$ (perhaps because we are concentrating on situations where $P$ is true).
Then we are justified claiming that $Q$ is also true. 

Some declarative sentences are not statements. 
``$x<y$'' can not be interpreted as being true or false because the values of $x$ and $y$ are not fixed.
On the other hand, ``When $x=7$ and $y=5$, $x<y$'' does have a definite meaning.

A \newterm{formula} is a sentence with variables appearing in it. 

 So ``implies'' plays an important role in mathematics. A statement ``If $P$ then $Q$'' would tell us that in any situation where $P$ is true, $Q$ is also true. It does not tell us anything about $Q$ in situations where $P$ is not true.
For example, ``if there is smoke, then there is fire'' makes a claim about the sitations where there is smoke. Namely, there is also fire. It does not tell us anything about situations where there is no smoke.


