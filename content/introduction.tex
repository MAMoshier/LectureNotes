\chapter{What's it all about}

Abstract things very often can stand in for, or \emph{represent}, concrete practical things while stripping away unwelcome clutter. For example, it is certainly useful to know that addition is a commutative operation ($x+y = y+x$ is always true no matter what numbers $x$ and $y$ happen to be). It does not matter what $x$ and $y$ represent (the number of math and the number of computer science students, or the length of two sticks, or the sizes of two bank accounts). The fact is we can think about addition \emph{abstractly} without needing to know what the numbers stand for. This is incredibly useful, and is at the heart of mathematics.

Multiplication is also commutative. 
This, surely, is also a useful thing to know. 
But what does it mean to multiply the number of math stduents by the number of computer science students? 
Concretely, it is hard to see what that could mean (whereas adding them seems obviously to represent  gathering both types of student into a single collection). 
Nevertheless, the abstract operation of multiplication makes sense. 
And it is commutative.

We now have two operations (addition, multiplication) that are quite different from one another, but they share some properties: both are commutative. You can easily think of other properties they both have ($x+(y+z)=(x+y)+z$ and $x(yz)=(xy)z$, for example). You can probably also come up with some properties that they do not share (hint: Does every number have a negative? Does every number have a reciprocal?). So thinking about arithmetic in abstract terms helps us understand where the abstractions are appropriate.

Actually, the answers to most questions depend on further levels of abstraction. For example, whether or not every number has a negative depends on what sort of numbers we care about. If we only want to think about counting numbers -- $0$, $1$, $2$, and so on -- then the only number that has a negative is $0$ because $0+0=0$, but for example ``$-1$'' does not even exist as far as we are concerned.

For another example, the claim that every number has a square root is obviously false if our numbers are the real numbers since negative numbers do not have square roots. But it happens that every complex number $a+bi$ does have exactly two square roots (except $0+0i$ which has only itself as a root). And though a formula for the square root is not easy to find, the property that all complex numbers have two square roots (except for $0$) is part of what makes complex numbers useful.	
So we can not just think about things like addition, multiplication and so on, without also thinking about the type is data we are concerned with.

A big advantage of dealing in abstraction is that, by separating mathematical \emph{ideas} from any concrete interpretation, we sometimes discover other completely novel uses. Try the following procedure: 
\begin{enumerate}
	\item Write down your credit card number (I will not ask you to give anhthing away). For example, suppose the number is $845937493485$ (not a real card). 
	\item Starting from the 2nd digit on the right, double every other digit and leave the other digits as is. For our example, the result would be $16 4 10 9 6 7 8 9 6 4 16 5$. I have put spaces in here to make it more readable.
	\item Now add all the individual digits of the result. That is, count $16$ as $1+6$. For the example, the result is $1+6+4+1+0+9+6+7+8+9+6+4+1+6+5 = 73$. 
\end{enumerate}
Your result will be a number that ends with $0$, assuming you have a valid credit card.
Now try the same procedure using your credit card number with a mistake. Try swapping two adjacent digits, or change any single digit.
The result will not end in $0$ (with only one exception -- swapping adjacent $9$ with $0$ does not do anything to the result).

This procedure (called \emph{Luhn's checksum}) uses the simple arithmetic of adding digits and doubling some of digits to test that a number is, at least possibly, a valid card number. 
It catches the easiest errors (adjacent digit swaps and single digit changes).
But this has nothing to do with the usual concrete interpretation of addition, as in $4$ apples plus $3$ apples give us $7$ apples.
Arithmetic is used as purely abstract operations on these very large numbers, but the numbers themselves do not represent counting anything, or measuring anything. 

In this course, we deal head on with mathematics as the study of abstract stucture.
This can be frustrating at first because the concrete applications are not always obvious.
The pay off comes when we see that abstraction from particulars leads to much wider applications than we could have anticipated.

We have claimed that addition and multiplication are commutative.
But is that really true? Is it the case that $1239283 + 11^{11^11}$ (a number with more that $100$ trillion digits) is the same as $11^{11^11} + 1239283$?
These numbers are far too big to check explicitly.
But we are completely confident that thet two numbers are equal.
Why? 

We should be able to convince ourselves that $m+n = n+m$ is true for any two counting numbers, not just take it as being ``obvious''. 
Indeed, figuring out how to convince ourselves (and other mathematicians) about what is true is one of the main activities of mathematics.
In a sense, this makes mathematics a very human activity.
It is mainly concerned with persuasion:of the sort: I know that addition is commutative, here is why you should know it too \ldots

The standards for persuasion in mathematics are high. We do not settle for ``preponderance of evidence'' or ``beyond reasonable doubt''.
We aim for ``beyond any doubt''. 
In most human endeavours (including in the sciences), that standard would be paralyzing, but in mathematics, where abstract structure is the object of investigation, it is not only within reach, but it a standard that makes sense.

In mathematics, a \newterm{proof} is a convincing argument.
We will spend a great deal of our time discussing and constructing proofs.
Proofs that meet the standards of mathematics have a vocabulary, grammar and prosody of their own.
That is, there are specific word usages that have precise technical meanings (vocabulary).
There are basic rules of proof construction (grammar).
And there are general ways to make a proof more understandable, and therefore more convincing (prosody).
Our first lecture presents a brief introduction to proofs.


