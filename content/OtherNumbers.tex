\chapter{Other Number Systems}\label{lec:OtherNumbers}

Discrete Mathematics is not only concerned with natural numbers. Integers, rational numbers, real numbers and even, on a few occasions, complex numbers, all play their roles. In this lecture, we discuss, with much less formality, integers and rational numbers, leaving real and complex numbers to a courses on analysis.

\begin{goals}
\begin{goal}{Lecture}
\item Introduce the standard notation for these number systems.
\item Discuss briefly the extensions of arithmetic to integers, rationals.
\item Wave hands and mumble unconvincingly about real and complex numbers.
\end{goal}

\begin{goal}{Study}
\item Learn the meanings of the symbols for standard number systems.
\item Prove some simple facts about arithmetic on integers and rational numbers.
\end{goal}
\end{goals}

\section{Integers}

Our intention here is to give a general overview of the integers in light of our detailed study of natural numbers.
Though we could, we will not go into the same level of detail.

Integers, we understand, include the natural numbers but also include negatives. They constitute the simplest possible extension of the natural numbers in such a way so that for any integers $a$ and $b$, there is a solution of the equation $a = b + x$.  
So the fundamental structure of integers is that (a) the natural numbers are integers and (b) addition on integers is invertible, i.e., subtraction works correctly.

\begin{postulate}\label{post:IntSignature}
The integers have the following basic structure.

\begin{itemize}
\item Every natural number is an integer. For this discussion, we will use letters $a$, $b$, $c$, \ldots for integers,
and $m$, $n$, \ldots for natural numbers.
To emphasize when a natural number $m$ plays the role of an integer we write ${}^+m$.
\item For every two integers, $a$ and $b$, there is a \emph{sum} and a \emph{difference}, denoted by $a+b$ and $a-b$. 
\end{itemize}
\end{postulate}

We must insist that addition on integers behaves as we should expect. 
\begin{postulate}\label{post:IntAddition}
For natural numbers $m$ and $n$,
\[{}^+m + {}^+n = {}^+(m+n).\]
That is, adding $m$ and $n$ \emph{as integers} is the same as adding them \emph{as natural numbers}.

Furthermore, addition on integers is associative and commutative, and $a + {}^+0 = a$ for all integers $a$.
\end{postulate}

Next, sum and difference are related.

\begin{postulate}\label{post:IntSubtraction}
For any integers $a$, $b$ and $c$, 
\[a = b + c\quad\text{if and only if}\quad a - b = c\]
\end{postulate}

Finally, we require minimality.

\begin{postulate}\label{post:IntInduction}
  No integers can be removed without violating Postulate \ref{post:IntSignature}.
\end{postulate}

We do not go through the details, but these axioms determine the structure of the integers similar to the way our earlier axioms determine the structure of the natural numbers. 
Moreover, all the familiar laws of addition and subtraction follow.
We illustrate with two simple laws. 
As usual, we can abbreviate ${}^+0-a$ by $-a$.

\begin{lem}\label{lem:IntAdditiveInverse}
For any integer $a$, ${}^+0= a + -a$.

\begin{proof}
  Since ${}^+0-a = {}^+0-a$ (trivially), Postulate \ref{post:IntSubtraction} yields ${}^+0 = a + ({}^+0-a) = a + (-a)$.
\end{proof}
\end{lem}

\begin{lem}\label{lem:IntAddSub}
  For integers $a$ and $b$, $a-b = a + (-b)$.

  \begin{proof}
    $a = a + {}^+0 = a + (-b + b) = b + (a+(-b))$ by the basic laws of addition (identity, commutativity and associativity)
and the previous lemma. So by Postulate \ref{post:IntSubtraction}, $a-b = a + (-b)$.
  \end{proof}
\end{lem}

In the axiomatization of integers, we did not mention multiplication. This is because multiplication is definable from addition. But we need an aditional fact.

Say that an integer $a$ is ``simple'' if either $a={}^+m$ or $-a = {}^+m$ for some natural number $m$. 

\begin{lem}
	Every integer is simple.
	
	\begin{proof}
		Exercise
	\end{proof}
\end{lem}

\begin{defn}
For any two integers $a$ and $b$, the \emph{product} $a\cdot b$ defined as follows:
\begin{align*}
	a\cdot {}^+0 &= {}^+0\\
	a\cdot {}^+(k^\nxt) &= a + a\cdot {}^+k\\
	a\cdot -({}^+m) &= -(a\cdot {}^+m)
\end{align*}
\end{defn}

These requirements are enough to ensure that multiplication is the operation we expect it to be.
We will not verify that fact, but it involves an inductive argument similar to the arguments that we saw in previous lectures.

\begin{exer}
	\begin{exercise}
\item Show that every integer is simple, using only the postulates given here.

Hint: First observe that, trivially, every integer of the form ${}^+m$ is simple.
This acts as the basis for an inductive proof. 
Now suppose $a$ and $b$ are simple (the inductive hypothesis).
Show that both $a+b$ and $a-b$ are simple (the inductive step).
Since the integers are minimal for including the natural numbers and being closed under $+$ and $-$, this shows that all integers are simple.
\item Using only the axioms for integers, the foregoing two lemmas and the definition of multiplication, prove the following (remember
that $-a$ abbreviates ${}^+0 - a$). 
You should prove these in the order presented here, as later exercises may depend on earlier ones. For all integers $a$, $b$ and $c$ show that the following are true.
\begin{enumerate}
\item Addition is cancellative: $a + c = b + c$ implies $a = b$.
\item $a-b = -(b-a)$
\item $(a+b) - c = a + (b-c)$
\item $a-(b+c) = (a-b)-c$
\item $-(-a) = a$
\item ${}^+0\cdot a = {}^+0$
\item $-a\cdot -b = a\cdot b$
\item Multiplication distributes over addition.
\end{enumerate}
\end{exercise}
\end{exer}

\section{Rational Numbers}

Rational numbers, roughly, are minimal with respect to the integers and inverting multiplication, as 
integers are minimal with respect to natural numbers and inverting addition. Care must be taken, however, because ${}^+0\cdot a = {}^+0\cdot b$ is always true, this does not tell us anything about $a$ and $b$. So ``division by $0$'' can not make sense.  

\begin{postulate}\label{post:RatSignature}
The rational numbers have the following basic structure.
\begin{itemize}
\item Every integer is a rational number. For this discussion, to emphasize when an integer $a$ plays the
role of a rational number, we write $\frac{a}{1}$. 
\item For every two rational numbers $p$ and $q$, there is a \emph{product}, denoted by $p\cdot q$.
Moreover, if $q\neq \frac{0}{1}$, then there is a \emph{quotient}, denoted by $p/q$.
\end{itemize}
\end{postulate}

In analogy with integers, where addition and subtraction are the main operations, we insist that
multiplication behaves correctly. 

\begin{postulate}\label{post:RatMultiplication}
For integers $a$ and $b$,
\[\frac{a}1 \cdot \frac{b}1 = \frac{ab}1\]
That is, multiplying integers $a$ and $b$ \emph{as rational numbers} is the same as multiplying them \emph{as integers}.

Furthermore, multiplication is associative and commutative, and $p \cdot \frac{1}{1} = p$ for all rational numbers $p$.
\end{postulate}

Again, continuing the analogy with integers, multiplication and division must cooperate.
\begin{postulate}\label{post:rationaldivision}
For any rationals $p$, $q$ and $r$,
if $q\neq \frac{0}1$, then 
\[p= q \cdot r\quad\text{if and only if}\quad p/q = r\]
\end{postulate}

Finally, we require minimality.

\begin{postulate}\label{post:RationalInduction}
  No rational numbers can be removed without violating Postulate \ref{post:RatSignature}.
\end{postulate}

Like the axioms for integers, these axioms completely determine the rational numbers along with multiplication and division. 
To extend addition and subtraction to the rational numbers we can require the following.

For integers, we introduced the idea of \emph{simplicity} to capture the fact that every integer is either a natural number or a negated natural number. The analogous idea for rational numbers is that every rational number is expressible as a fraction.

\begin{defn}
	Say that a rational number $p$ is \newterm{fractional} if it is the case that $p=a/b$ for some integer $a$ and non-zero integer $b$. 
\end{defn}

\begin{lem}
	Every rational number is fractional.
	
	\begin{proof}
		Like the proof that every integer is simple, this involves an inductive argument. Namely, every integer is, according to the basic structure of rationals, fractional. Namely, $a/{}+1 = \frac{a}1$. Indeed the difference between the two notations is almost trivial.
		For the sake of completeness though, note that an integer $a$ is written as $\frac{a}{1}$ only to emphasize that it is also a rational number. So $a/{}^+1 = a$ because $a = {}^+1\cdot a$.
		
		Suppose $p$ and $q$ are fractional. So $p=a/b$ and $q=c/d$ for some integers $a$, $b$, $c$ and $d$ with $b\neq 0$ and $d\neq 0$. Note that $a/b = p$ means that $a=b\cdot p$. Likewise $c=d\cdot q$. We claim that $p\cdot q = (a\cdot c)/(b\cdot d)$. This is true if
		$p\cdot q \cdot b\cdot d = a\cdot c$. But this follows from the foregoing. Since $b$ and $d$ are not zer, $b\cdot d$ is not zero. Hence $p\cdot q$ is fractional.
		We also claim that $p/q$ is fractional. Here we need two cases. Suppose $c=0$. Then $p=0/1$ as needed. Suppose $c\neq 0$. 
		Then $p/q = (a\cdot d)/(b\cdot c)$ because $a\cdot d \cdot q = p b\cdot c\cdot p$. We leave it to the reader to figure out why this last equation is true.
	\end{proof}
\end{lem}

Notice that this lemma does not tell us anything about simplified fractions. It merely says that every rational number can be written as a fraction. But over course, $2/4 = 4/8 = 6/12$ and so on.

\begin{defn}
For any two rational numbers $p = a/b $ and $q=c/d$, their \emph{sum}, denoted by $p+q$ is the rational number $(ad+cb)/bd$.
\end{defn}

This definition is not entirely legitimate yet. Because it could be the case that the result $(ad+cb)/bd$ depends on how we write $p$ and $q$ as fractions. In fact, though, this is not the case. For suppose $a/b = a'/b'$
and $c/d=c'/d'$. Then $(ad+cb)/bd = (a'd'+c'b')/b'd'$. So the way we add \emph{fractions} is correct for adding rational numbers.
These requirements are enough to ensure that addition is the operation we expect it to be. 

Now suppose $a$ is an integer and $m$ is a positive natural number. Then we
can write $\frac{a}m$ as an abbreviation of $\frac{a}{1} / \frac{{}^+m}1$. This is
the usual \emph{fraction} notation for rational numbers.
For a non-zero rational number $p$, we also usually write $p^{-1}$ to abbreviate 
$\frac{{}^+1}{1}/ p$.

\begin{exer}
\begin{exercise}
\item Using the proof of Lemma \ref{lem:IntAdditiveInverse} as a prototype, show that if $p\neq 0$,
then $p\cdot p^{-1} = 1$. 
\item Using the proof of Lemma \ref{lem:IntAddSub} as a prototype, show that if $q\neq \frac{{}^+0}{1}$ then $p/ q = p\cdot q^{-1}$.
\item Just using our axioms and the fraction notation (and the previous two exercises),
show that $\frac{4}{6} = \frac{2}{3}$.
\end{exercise}
\end{exer}

\section{Real and Complex Numbers}

We will not deal with rigorous characterizations of the real numbers and complex numbers in this course. 
A first course in real analysis is the appropriate place for such considerations. 
We simply take for granted that $\RR$ constitutes a suitable
extension of $\QQ$ to account for \emph{limits} (the mathematics you know from calculus). Likewise, $\CC$
constitutes a suitable extension of $\RR$ to account for 
algebraic closure (all polynomials having roots). The main things to know are that every rational number is real,
every real number is complex, and arithmetic operations behave as they should for real and complex numbers.

\begin{exer}
\begin{exercise}
  \item Relax. Take a walk.
\end{exercise}
\end{exer}